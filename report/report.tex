\documentclass[11pt]{article}
\usepackage{geometry}
\geometry{margin=0.95in}
\usepackage{booktabs}
\usepackage{adjustbox}
%\usepackage{emoji}
%\usepackage{biblatex}
%\bibliographystyle{plain}
%\bibliography{lib}


\title{\vspace{-1.5cm}IDS703 Final Project Report}
\author{Anna Dai, Satvik Kishore, Moritz Wilksch}
\date{December 12th, 2021}

\begin{document}
\maketitle

% --------------------------------------------------------------------------------
\section{Introduction}

In this project, we have worked on tweet classification as a Natural Language Processing problem, more specifically, as a document classification problem. Twitter is a microblogging service where users post publicly visible "tweets", that are essentially texts with less than 280 characters. These tweets may also contain other media objects which are discarded for the purposes of this project. These tweets most often serve as discussion pieces as part of larger conversations. They are relevant to any number of topics under discussion. These "topics" are also often explicitly highlighted by the user using a "hashtag", i.e. text with '\#' followed by the topic name, or a commonly used shorthand for it. In our project, we treat these hashtags as "topics" for our document classification model, where each tweet is an instance of a document.

% --------------------------------------------------------------------------------
\section{Data}
We have manually selected 7 topics, or hashtags for classification: crypto, tesla, championsleague, formula1, thanksgiving, holidays, covid19.

These topics were intentionally selected to have topics that have some amount of overlap between them (holidays and thanksgiving) while also having topics easier to differentiate (crypto vs formula1). We scraped data using the python library twint \cite{twint}, scraping approximately 10,000 tweets for each of the seven topics. We also applied a few pre-processing steps before the next steps. This included tokenization of the tweets into words. [TODO: elimination of common words?]. Conversion of emojis into tokens (each appearance of an emoji is a token. Multiple emojis strung together are treated as different tokens in sequence). We also removed punctuation marks. Following these steps, we split our data into three parts using a 60:20:20 split to form a training dataset, a validation dataset, and a test dataset. 


% --------------------------------------------------------------------------------
\section{Methodology}
\subsection{Workflow}
The collected corpus of almost 70,000 tweets is randomly split into a training, validation, and test set in a 60/20/20 ratio. We will train models on the train set, find optimal hyperparameters using the validation set, and report all performance metrics in the \emph{Results} section on the test set, which is not used for any other purpose than evaluating each model once.


\subsection{Generative Model}
We use a Latent Dirichlet Allocation (LDA) model as a generative model to learn from the corpus we have collected. This type of model does not require any hyperparamter tuning, and thus is trained using a combination of the training and validation dataset. We used the LDA implementation from scikit-learn \cite{sklearn}. It is fit to the corpus to find seven separate topics, as this is the number of actual topics in the training set. Subsequently, for each inferred topic, we manually inspect the top 50 words that are associated with it to assign it a name. This manual step is necessary, as the order of topics is not preserved. In fact, the LDA model does not even guarantee to find the same topics that were collected in the original data set. This shortcoming will be discussed in the \emph{Results} and \emph{Conclusion} sections. 
To use the LDA for document classification,  we let it infer the topic distribution of each document in the test set and use the $argmax(\cdot)$ function to assign each document the topic that is most prevalent according to the LDA.
Finally, the LDA is used to generate synthetic data. For each topic, we sample 10,000 artificial documents, the length of which is sampled from the empirical distribution of tweet lengths each time. Similar to the actual data set, the synthetic data set is also split into a train, validation, and test set.


\subsection{Discriminative Model}
- network description
- hyperparameter tuning on synth data
- application to real
- transfer learning on real data

- compare to model that has ONLY been trained on real?

% --------------------------------------------------------------------------------
\section{Results}
\subsection{Benchmarking on Synthetic Data}

\begin{center}
\begin{tabular}{lrrrr}
\toprule
{} &  precision &  recall &  f1-score &  support \\
\midrule
thanksgiving    &      0.916 &   0.969 &     0.942 &     1979 \\
formula1        &      0.986 &   0.897 &     0.939 &     2172 \\
covid19         &      0.960 &   0.966 &     0.963 &     2002 \\
championsleague &      0.957 &   0.965 &     0.961 &     1926 \\
crypto          &      0.945 &   0.901 &     0.923 &     2017 \\
tesla           &      0.943 &   0.985 &     0.963 &     1967 \\
holidays        &      0.924 &   0.954 &     0.939 &     1937 \\
\bottomrule
\end{tabular}

	
\end{center}
Table 1: Benchmark results of neural net (trained on synthetic data only) on synthetic data


\subsection{Benchmarking on Real Data}

\begin{center}
\begin{tabular}{lrrrr}
\toprule
{} &  precision &  recall &  f1-score &  support \\
\midrule
thanksgiving    &      0.548 &   0.359 &     0.434 &     2357 \\
formula1        &      0.023 &   0.032 &     0.027 &     1420 \\
covid           &      0.191 &   0.416 &     0.261 &      911 \\
championsleague &      0.601 &   0.521 &     0.558 &     2276 \\
crypto          &      0.689 &   0.353 &     0.467 &     3894 \\
tesla           &      0.336 &   0.490 &     0.399 &     1388 \\
holidays        &      0.102 &   0.152 &     0.122 &     1423 \\
\bottomrule
\end{tabular}

\end{center}
Table 2: Benchmark results of neural net (trained on synthetic data only) on real data



\begin{center}
\begin{tabular}{lrrrr}
\toprule
{} &  precision &  recall &  f1-score &  support \\
\midrule
thanksgiving    &      0.386 &   0.524 &     0.445 &     1539 \\
formula1        &      0.009 &   0.012 &     0.010 &     1468 \\
covid19         &      0.797 &   0.518 &     0.628 &     3100 \\
championsleague &      0.857 &   0.542 &     0.664 &     3071 \\
crypto          &      0.543 &   0.751 &     0.630 &     1389 \\
tesla           &      0.838 &   0.567 &     0.676 &     3036 \\
holidays        &      0.032 &   0.164 &     0.054 &      397 \\
\bottomrule
\end{tabular}

\end{center}
Table 3: Benchmark results of neural net (trained on synthetic data and real data) on synthetic data

\begin{center}
\begin{tabular}{lrrrr}
\toprule
{} &  precision &  recall &  f1-score &  support \\
\midrule
thanksgiving    &      0.814 &   0.921 &     0.864 &     1362 \\
formula1        &      0.742 &   0.889 &     0.809 &     1691 \\
covid19         &      0.811 &   0.798 &     0.804 &     2021 \\
championsleague &      0.878 &   0.673 &     0.762 &     2578 \\
crypto          &      0.764 &   0.782 &     0.772 &     1950 \\
tesla           &      0.798 &   0.709 &     0.751 &     2273 \\
holidays        &      0.824 &   0.974 &     0.893 &     1794 \\
\bottomrule
\end{tabular}

\end{center}
Table 4: Benchmark results of neural net (trained on synthetic data and real data) on real data

\begin{center}
\begin{tabular}{lrr}
\toprule
{} &  precision &  recall \\
\midrule
thanksgiving    &      0.395 &   0.482 \\
formula1        &      0.052 &   0.046 \\
covid19         &      0.519 &   0.350 \\
championsleague &      0.616 &   0.708 \\
crypto          &      0.566 &   0.553 \\
tesla           &      0.636 &   0.546 \\
holidays        &      0.048 &   0.092 \\
\bottomrule
\end{tabular}

\end{center}
Table 5: Benchmark results of neural net (trained on real data only) on synthetic data


\begin{center}
\begin{tabular}{lrr}
\toprule
{} &  precision &  recall \\
\midrule
thanksgiving    &      0.885 &   0.822 \\
formula1        &      0.825 &   0.810 \\
covid19         &      0.865 &   0.769 \\
championsleague &      0.796 &   0.844 \\
crypto          &      0.862 &   0.710 \\
tesla           &      0.680 &   0.855 \\
holidays        &      0.838 &   0.979 \\
\bottomrule
\end{tabular}

\end{center}
Table 6: Benchmark results of neural net (trained on real data only) on real data


\begin{center}
\begin{tabular}{lrr}
\toprule
{} &  precision &  recall \\
\midrule
thanksgiving    &      0.946 &   0.979 \\
formula1        &      0.986 &   0.977 \\
covid19         &      0.986 &   0.986 \\
championsleague &      0.986 &   0.969 \\
crypto          &      0.986 &   0.987 \\
tesla           &      0.986 &   0.976 \\
holidays        &      0.967 &   0.969 \\
\bottomrule
\end{tabular}
\end{center}
Table 7: Benchmark results of LDA classification on synthetic data

\begin{center}
\begin{tabular}{lrr}
\toprule
{} &  precision &  recall \\
\midrule
thanksgiving    &      0.150 &   0.107 \\
formula1        &      0.558 &   0.562 \\
covid19         &      0.034 &   0.022 \\
championsleague &      0.339 &   0.443 \\
crypto          &      0.333 &   0.304 \\
tesla           &      0.412 &   0.319 \\
holidays        &      0.387 &   0.670 \\
\bottomrule
\end{tabular}
% Accuracy: 0.34011266369156484
\end{center}
Table 8: Benchmark results of LDA classification on real data

% --------------------------------------------------------------------------------
\section{Conclusion}
\begin{adjustbox}{width=\columnwidth,center}
\begin{tabular}{lllllll}
\toprule
    hamilton &   barcelona &     vaccine & christmas &     turkey &         elon &         btc \\
\midrule
         max & bayernbarça &    immunity &  december &          🦃 &         musk &     cryptos \\
        ocon &           ⚽ &    vaccines &  nicholas &       nick &          fsd &       coins \\
       lewis &       zenit &      deaths &   festive &     cotton &       teslas &         eth \\
         vsc &       barca &       tests &     visit &       flex &           ev &      ssfeed \\
championship &      bayern &    measures &   snowman &  gratitude & supercharger &      tether \\
         abu &         ucl &       boris &      pack &        nov &      binance &          io \\
         fia &   liverpool &     booster &    wreath &   rosemary &    cointrade & opportunity \\
     formula &       milan &  vaccinated &   rainbow & chronicles &         giga &        dump \\
        race &    matchday & vaccination &         🥃 &  skyrocket &    autopilot &   analyzing \\
\bottomrule
\end{tabular}

\end{adjustbox}




\end{document}
